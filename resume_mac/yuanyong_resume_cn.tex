%!TEX program = xelatex
\documentclass[11pt,a4paper]{moderncv}

%主题设置
\moderncvstyle{classic}                        % 选项参数是 ‘casual’, ‘classic’, ‘oldstyle’ 和 ’banking’
\moderncvcolor{blue} % optional argument are 'blue' (default), 'orange', 'red', 'green', 'grey' and 'roman' (for roman fonts, instead of sans serif fonts)

\usepackage{fontspec,xunicode,xltxtra}
% 设置英文字体
\defaultfontfeatures{Scale=MatchLowercase,Scale=0.82} % 设置字体大小,Scale=0.8等
\setmainfont{Hiragino Sans GB} %主字体
\setsansfont{Hiragino Sans GB} %无衬线字体
\setmonofont{Hiragino Sans GB} %等宽字体
% 设置中文字体
\usepackage{xeCJK}
% Scheme: {STFangsong}, 冬青黑体简体中文{Hiragino Sans GB W3},
\setCJKmainfont{Hiragino Sans GB W3}
\setCJKsansfont{Hiragino Sans GB W3}
\setCJKmonofont{Hiragino Sans GB W3}

% 设置最后一次更新时间
\usepackage{lastpage}
\usepackage{fancyhdr}
\pagestyle{fancy}
\fancyhf{}
\fancyfoot[L]{\footnotesize\textit{袁勇个人简历MacVersion}}
\fancyfoot[R]{\footnotesize\textit{最后一次更新于: \today}}

% 调整页边距
\usepackage[scale=0.9]{geometry}
\setlength{\hintscolumnwidth}{3cm}						% if you want to change the width of the column with the dates
%\AtBeginDocument{\setlength{\maketitlenamewidth}{6cm}}  % only for the classic theme, if you want to change the width of your name placeholder (to leave more space for your address details
%\AtBeginDocument{\recomputelengths}                     % required when changes are made to page layout lengths

% 在其他左侧插入标志
\usepackage{manfnt}
\newcommand{\hello}{{\tiny\textdbend}}

%\usepackage{xcolor}
\linespread{1.2}

% 设置超链接字体颜色
\AtBeginDocument{
    \hypersetup{colorlinks,urlcolor=blue}
}
%\hypersetup{%
%  colorlinks=true,% hyperlinks will be black
%  linkbordercolor=red,% hyperlink borders will be red
%  pdfborderstyle={/S/U/W 1}% border style will be underline of width 1pt
%}

% 设置个人信息
%\firstname{}
%\familyname{袁勇}
\name{\textbf{袁}}{\textbf{勇}}
\title{\textbf{湖北|男}}                              % optional, remove the line if not wanted
\address{}{北京市海淀区}                       % optional, remove the line if not wanted
\mobile{150-2955-2208}                         % optional, remove the line if not wanted
%\phone{phone (optional)}                      % optional, remove the line if not wanted
%\fax{fax (optional)}                          % optional, remove the line if not wanted
\email{willard.yuan@gmail.com}                 % optional, remove the line if not wanted
\homepage{yongyuan.name}                       % optional, remove the line if not wanted
%\extrainfo{♂34岁}                             % optional, remove the line if not wanted
\photo[64pt]{qr}                               % '64pt' is the height the picture must be resized to and 'picture' is the name of the picture file; optional, remove the line if not wanted
\social[github]{willard-yuan}
\quote{\textit{专注计算机视觉领域问题}}

% to show numerical labels in the bibliography; only useful if you make citations in your resume
%\makeatletter
%\renewcommand*{\bibliographyitemlabel}{\@biblabel{\arabic{enumiv}}}
%\makeatother

% bibliography with mutiple entries
%\usepackage{multibib}
%\newcites{book,misc}{{Books},{Others}}

%\nopagenumbers{}                             % uncomment to suppress automatic page numbering for CVs longer than one page
%----------------------------------------------------------------------------------
%            content
%----------------------------------------------------------------------------------
\begin{document}
\maketitle
%\vspace{-3em}      %缩小段落的间距

\section{\textbf{教育背景}}
\cventry{2013.9 - 2016.6}{硕士学位}{中国科学院大学}{信号与信息专业}{保研}{}
\cventry{2009.9 - 2013.6}{学士学位}{西安电子科技大学}{电子信息科学与技术专业}{专业 top 3\%}{}
%\vspace{-1em}

\section{\textbf{工作经历}}
\cventry{2016.12 - 至今}{快手}{}{MMU}{算法工程师}{
\begin{itemize}
\item 设计开发了一种视频相似重排与视频查重校验方法.
\item 设计开发了基于OCR文字识别、Faster RCNN两套Logo识别系统,更新维护基于局部特征匹配Logo识别系统。检测准确率:99.5+\%,检测召回量: 640万视频评估召回70万Logo
\item 研发基于CNN和传统特征融合的视觉检索系统,解决相似检索和物体检索视觉搜索问题。平均检索精度:Oxford Building数据集mAP取得80\%(待上线)
\item 设计开发了快手app视频截屏检测器,检测准确率100\%,650万视频评估召回2万快手app截屏视频
\item 设计开发了视频查重校验器,作为对局部特征查重校验器的补充,增加了查重的召回率,600万上评估在原局部特征查重校验器基础上额外召回3720对重复视频。
\end{itemize}}

\cventry{2016.7 - 2016.12}{美团}{}{外卖风控部}{算法开发}{
\begin{itemize}
\item 设计开发了新的商户抓取策略,维护、优化已有的商户抓取逻辑,对新美大外卖业务的数据做日常的分析、监控及报表.
\end{itemize}}

\cventry{2015.10 - 2016.6}{ETRACK眼控技术}{}{学生创业团队}{算法设计与开发}{
\begin{itemize}
\item 负责设计瞳孔检测与瞳孔中心检测算法并用C++实现,采用OpenMP实现多线程实时处理.
\item 负责设计瞳孔中心到屏幕坐标的映射方法并用C++实现,通过安卓NDK供JAVA调用.
\item 负责桌面版本的开发,使用QT框架构建图形界面, 人眼检测算法核心部分:\link[视频演示]{http://yongyuan.name/project/}.
\end{itemize}}

\section{\textbf{科研经历}}
\subsection{中科院西安光学精密机械研究所(2013--2016)}
\cventry{2013.3 - 2016.6}{基于内容的图像检索(CBIR)}{}{课题研究方向}{}{
\begin{itemize}
\item 熟练掌握BoW、VLAD、FV等特征编码方法,精通CBIR及大规模索引技术.
\item 掌握并积累机器学习中常用的降维、聚类、分类以及图像物体识别技术.
\item 提出并发表一种基于稀疏表达的哈希编码方法,详见\link[HABIR工具包主页]{http://yongyuan.name/habir/}.
\item 对同款物体的检索有较多的积累经验(衣服、鞋子等大型图像库30万);有对特定类图像诸如皮革、纺织图像等进行检索的经历;在13万量级的图库上做过广告logo的搜索.
\item 对人脸数据检索与识别、医学影像检索有相应的经历,并对深度学习(CNN卷积神经网络)具备一定的理解.
\end{itemize}}

\cventry{2015.1 - 2015.4}{基于卷积神经网络的CBIR演示原型系统PicSearch}{}{兴趣驱动型项目}{协作开发}{PicSearch是一个在线图像检索原型系统,使用了CNN卷积网络模型.
\begin{itemize}
\item 线下完成图像特征的提取,并做了一定的降维处理,后台在线特征匹配与排序用python实现,服务器采用了python轻量级web开发框架CherryPy,采用Boostrap框架优化前端交互界面.
\item 图库为包含29780张图片的Caltech-256公开数据集,采用特征常驻内存的方式进行了代码的优化,使其能及时地响应用户的查询请求(毫秒级),在线演示地址PicSearch: \link[search.yongyuan.name]{http://search.yongyuan.name/}(已下线),本地演示效果: \link[演示视频]{http://v.youku.com/v_show/id_XMTM0NzYyMzE4OA==.html?firsttime=0}.
\end{itemize}}

\cventry{2015.3 - 2015.7}{基于词袋模型的物体检索原型DupSearch系统}{}{兴趣驱动型项目}{独立开发}{DupSearch是一个针对Object Retrieval或Duplicate Search而写的图像检索原型系统.
\begin{itemize}
\item 在Oxford Building公开数据库上平均检索精度达到83.35\%,对于光照、旋转、视角等具有较好的适应性,在线匹配在服务器上能较快的响应查询,并且在不复杂化现有模型情况下仍有改进提高MAP的空间.
\item 图像库测试规模30万,取得了很不错的检索效果,算法原型系统已售予某公司,15万衣服库检索示例详见\href{https://github.com/willard-yuan/pkbigdata-image-search}{GitHub},此外,对于广告logo的搜索也能取得很高的检索精度.
\end{itemize}}

\cventry{2014.7 - 2015.5}{复杂低空飞行的自主避险理论与方法研究(973)}{}{项目参与者}{}{多源协同感知周围环境,对复杂低空环境中可能的危险障碍物进行实时检测,并完成飞行器的自主避险.
\begin{itemize}
\item 负责可见光传感器数据与激光雷达传感器点云数据的融合,消除高压线检测时的误检.
\item 负责桥梁、高压线塔、作为异常目标入侵的滑翔机等危险障碍物的实时检测.
\item 采用opencv、dlib等计算机视觉开源库,非电力线类障碍物检测采用HOG+SVM物体检测方法.
\end{itemize}}

\section{\textbf{出版物}}

\cventry{2016.1}{\textnormal{Xuelong Li}, \textbf{Yong Yuan} \textnormal{and Xiaoqiang Lu, Latent Semantic Minimal Hashing for Image Retrieval. IEEE TIP, 2016 (MINOR REVISION)} }{}{}{}{}
\cventry{2014.4}{\textbf{Yong Yuan}\textnormal{, Xiaoqiang Lu, and Xuelong Li. Learning Hash Functions Using Sparse Reconstruction. ACM ICIMCS, pp. 14-18, 2014 (Best Paper Runner-up Award)} }{}{}{}{}
\cventry{2014.6}{\textnormal{朱文涛,}\textbf{袁勇}\textnormal{. \href{http://www.ituring.com.cn/book/1349}{Python计算机视觉编程} (译作),图灵出版社} }{}{}{}{}
\cventry{2015.10}{\textnormal{李学龙,卢孝强,}\textbf{袁勇}\textnormal{. \href{https://www.google.com/patents/CN106033426A?cl=en&hl=zh-CN}{一种基于潜在语义最小哈希的图像检索方法} (专利)} }{}{}{}{}
%\vspace{-1em}

%\vspace{-1em}

\section{\textbf{开源项目}}
\cvline{2017.7 - Now}{构建CBIR领域传统特征与深度学习方法做图像检索的对比框架,详见\href{https://github.com/willard-yuan/cnn-cbir-benchmark}{GitHub.}}
\cvline{2016.8 - 2016.9}{以SeetaFaceEngine为基础,使用LSH索引技术构建了一个人脸检索系统,详见\href{https://github.com/willard-yuan/SeetaFaceLib}{GitHub.}}
\cvline{2015.4 - 2016.4}{基于MatConvNet以及VGGNet卷积神经网络模型构建的一个用于图像检索的实验工具包,详见\href{https://github.com/willard-yuan/CNN-for-Image-Retrieval}{GitHub.}}
\cvline{2013.2 - 2016.6}{整理并实现了一些流行的哈希算法及多种指标评价,目前该Matlab工具包已更新至V2.0,详见\href{https://github.com/willard-yuan/hashing-baseline-for-image-retrieval}{GitHub.}}
\cvline{2013.12 - 2014.6}{翻译《Programming Computer Vision with Python》时,为使读者更易于理解书中的内容,重新对书上的代码做了整理,并放在github 上,详见\href{http://yongyuan.name/pcvwithpython/}{项目主页.}}
\cvline{2014.2 - 2014.5}{基于稀疏重构的哈希编码方法的Matlab代码及检索指标评价,详见\href{https://github.com/willard-yuan/sparse-reconstruction-hashing}{GitHub.}}
%\vspace{-1em}

\section{\textbf{IT技能}}
\cvline{编程语言}{\small 会C++/C、OpenCV以及QT, 熟练Python, Matlab, SQL, 熟悉HTML, CSS, Spark}
\cvline{算法技能}{\small 精通CBIR(4年经历),熟练掌握深度学习与常见物体检测方法,对传统算法具备较好的理解}
\cvline{常用工具}{\small OS X、Linux、Caffe、OpenCV、Vim、Xcode、Dlib、Ipython Notebook、Git}
\cvline{GitHub}{\small {\link[github.com/willard-yuan]{https://github.com/willard-yuan}}}
%\vspace{-1em}      %缩小段落的间距

\section{\textbf{奖项}}
%\cventry{2014.07}{\textnormal {ICIMCS14最佳论文提名奖}}{}{}{}{}
%\cventry{2011--2012}{\textnormal {国家奖学金}}{}{}{}{}
%\cventry{2010--2011}{\textnormal {校内一等奖学金}}{}{}{}{}
%\cventry{2009--2010}{\textnormal {国家励志奖学金}}{}{}{}{}
\cvlistdoubleitem{中科院三好学生(2016.4)}{Best Paper Runner-up Award(2014.7)}
\cvlistdoubleitem{优秀学生巡回报告团成员(2012.12)}{国家奖学金(2012.11)}
\cvlistdoubleitem{校内一等奖学金(2011.11)}{国家励志奖学金(2010.11)}
%\vspace{-1em}      %缩小段落的间距

\section{\textbf{语言}}
\cvline{英语}{\small CET-6和\small CET-4,具备专业英文文献阅读、写作及翻译能力,平时会保持对CVPR等论文的阅读.}
%\cvlanguage{英语}{具备阅读专业英文文献能力及写作能力}{CET-6}
%\vspace{-1em}      %缩小段落的间距

\section{\textbf{其他}}
\cvline{\hello}{喜欢编码,热爱开源,有写\href{http://yongyuan.name/blog/}{博客}进行总结的习惯;具备较好的沟通、协调和组织能力.}
%\cvline{\hello}{热爱计算机视觉与互联网;喜欢编码,轻度代码洁癖,有写\href{http://yongyuan.name/blog/}{博客}进行总结的习惯;具有较好的人际沟通、协调和组织能力。}
%\cvlistitem{}

% Publications from a BibTeX file without multibib\renewcommand*{\bibliographyitemlabel}{\@biblabel{\arabic{enumiv}}}% for BibTeX numerical labels
%\nocite{*}
%\bibliographystyle{plain}
%\bibliography{publications}       % 'publications' is the name of a BibTeX file

% Publications from a BibTeX file using the multibib package
%\section{Publications}
%\nocitebook{book1,book2}
%\bibliographystylebook{plain}
%\bibliographybook{publications}   % 'publications' is the name of a BibTeX file
%\nocitemisc{misc1,misc2,misc3}
%\bibliographystylemisc{plain}
%\bibliographymisc{publications}   % 'publications' is the name of a BibTeX file

\end{document}


%% end of file `template_en.tex'.
