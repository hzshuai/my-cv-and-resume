%-------------------------------------------------------------------------------
%	SECTION TITLE
%-------------------------------------------------------------------------------
\cvsection{科研经历}


%-------------------------------------------------------------------------------
%	CONTENT
%-------------------------------------------------------------------------------
\begin{cventries}

%---------------------------------------------------------
  \cventry
    {基于内容的图像检索(CBIR)} % Job title
    {R.O.K Cyber Command, MND} % Organization
    {西安} % Location
    {Aug. 2013 - Exp. Apr. 2015} % Date(s)
    {
      \begin{cvitems} % Description(s) of tasks/responsibilities
        \item {精通CBIR 技术及其检索性能指标评价,熟练掌握了BoW 词袋模型、SIFT、VLAD、FV 等特征描述子.}
        \item {掌握并积累了机器学习中一些常用的降维手段、聚类算法、图像分类方法以及图像物体识别技术.}
        \item {深入研究过基于哈希的大规模图像检索技术,熟悉近几年来比较流行的哈希方法。针对一些流行的和经典的哈希方法进行了性能测试和指标评价,详见HABIR 工具包主页.}
      \end{cvitems}
    }

%---------------------------------------------------------
  \cventry
    {基于卷积神经网络的CBIR 演示原型系统PicSearch} % Job title
    {NEXON} % Organization
    {Seoul, S.Korea \& LA, U.S.A} % Location
    {Jan. 2013 - Feb. 2013} % Date(s)
    {
      \begin{cvitems} % Description(s) of tasks/responsibilities
        \item {线下完成图像特征的提取,并做了一定的降维处理,后台在线特征匹配与排序用python 实现,服务器采用了python 轻量级web 开发框架CherryPy,采用Boostrap 框架优化前端交互界面}
        \item {图库为包含29780 张图片的Caltech-256 公开数据集,采用特征常驻内存的方式进行了代码的优化,使其能及时地响应用户的查询请求(毫秒级),在线演示地址PicSearch:search.yongyuan.name(已下线),本地演示效果: 演示视频}
      \end{cvitems}
    }

%---------------------------------------------------------
  \cventry
    {基于词袋模型的物体检索原型DupSearch} % Job title
    {Undergraduate Research, Computer Vision Lab(Prof. Bohyung Han)} % Organization
    {Pohang, S.Korea} % Location
    {Sep. 2012 - Feb. 2013} % Date(s)
    {
      \begin{cvitems} % Description(s) of tasks/responsibilities
        \item {在oxford building 公开数据库上平均检索精度达到83.35\%,对于光照、旋转、视角等具有较好的适应性,在线匹配在服务器上能较快的响应查询,并且在不复杂化现有模型情况下仍有改进提高mAP 的空间.}
        \item {图像库测试规模达15 万,可以获得了很不错的检索效果,算法原型系统已售予某公司,15 万衣服库检索示例详见GitHub,此外,对于广告logo 的搜索也能取得很高的检索精度.}
      \end{cvitems}
    }

%---------------------------------------------------------
  \cventry
    {复杂低空飞行的自主避险理论与方法研究} % Job title
    {Software Maestro (funded by Korea Ministry of Knowledge and Economy)} % Organization
    {Seoul, S.Korea} % Location
    {Jul. 2012 - Jun. 2013} % Date(s)
    {
      \begin{cvitems} % Description(s) of tasks/responsibilities
        \item {负责可见光传感器数据与激光雷达传感器点云数据的融合,消除高压线检测时的误检.}
        \item {负责桥梁、高压线塔、作为异常目标入侵的滑翔机等危险障碍物的实时检测.}
        \item {使用了opencv、dlib 等计算机视觉开源库,非电力线类障碍物检测采用HOG+SVM 物体检测方法.}
      \end{cvitems}
    }

%---------------------------------------------------------
  \cventry
    {Software Engineer} % Job title
    {ShitOne Corp. (Start-up company)} % Organization
    {Seoul, S.Korea} % Location
    {Dec. 2011 - Feb. 2012} % Date(s)
    {
      \begin{cvitems} % Description(s) of tasks/responsibilities
        \item {Developed a proxy drive smartphone application which connects proxy driver and customer. Implemented overall Android application logic and wrote API server for community service, along with lead engineer who designed bidding protocol on raw socket and implemented API server for bidding.}
      \end{cvitems}
    }

%---------------------------------------------------------
  \cventry
    {Freelance Penetration Tester} % Job title
    {SAMSUNG Electronics} % Organization
    {S.Korea} % Location
    {Sep. 2013, Mar. 2011 - Oct. 2011} % Date(s)
    {
      \begin{cvitems} % Description(s) of tasks/responsibilities
        \item {Conducted penetration testing on SAMSUNG KNOX, which is solution for enterprise mobile security.}
        \item {Conducted penetration testing on SAMSUNG Smart TV.}
      \end{cvitems}
      %\begin{cvsubentries}
      %  \cvsubentry{}{KNOX(Solution for Enterprise Mobile Security) Penetration Testing}{Sep. 2013}{}
      %  \cvsubentry{}{Smart TV Penetration Testing}{Mar. 2011 - Oct. 2011}{}
      %\end{cvsubentries}
    }

%---------------------------------------------------------
\end{cventries}
